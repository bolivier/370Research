\documentclass[12pt]{article}

% Users of the {thebibliography} environment or BibTeX should use the
% scicite.sty package, downloadable from *Science* at
% www.sciencemag.org/about/authors/prep/TeX_help/ .
% This package should properly format in-text
% reference calls and reference-list numbers.

\usepackage{scicite}

% Use times if you have the font installed; otherwise, comment out the
% following line.

\usepackage{times}
\usepackage{url}
\usepackage{graphicx}
\usepackage{indentfirst}


\bibliographystyle{Science}

% image stuff
\graphicspath{ {data/} }



% The preamble here sets up a lot of new/revised commands and
% environments.  It's annoying, but please do *not* try to strip these
% out into a separate .sty file (which could lead to the loss of some
% information when we convert the file to other formats).  Instead, keep
% them in the preamble of your main LaTeX source file.


% The following parameters seem to provide a reasonable page setup.

\topmargin 0.0cm
\oddsidemargin 0.2cm
\textwidth 16cm 
\textheight 21cm
\footskip 1.0cm


%The next command sets up an environment for the abstract to your paper.

\newenvironment{sciabstract}{%
\begin{quote} \bf}
{\end{quote}}


% If your reference list includes text notes as well as references,
% include the following line; otherwise, comment it out.

\renewcommand\refname{References and Notes}

% The following lines set up an environment for the last note in the
% reference list, which commonly includes acknowledgments of funding,
% help, etc.  It's intended for users of BibTeX or the {thebibliography}
% environment.  Users who are hand-coding their references at the end
% using a list environment such as {enumerate} can simply add another
% item at the end, and it will be numbered automatically.

\newcounter{lastnote}
\newenvironment{scilastnote}{%
\setcounter{lastnote}{\value{enumiv}}%
\addtocounter{lastnote}{+1}%
\begin{list}%
{\arabic{lastnote}.}
{\setlength{\leftmargin}{.22in}}
{\setlength{\labelsep}{.5em}}}
{\end{list}}


% Include your paper's title here

\title{An Evaluation of Micro-blogging as a Regional Mental Health Indicator} 


% Place the author information here.  Please hand-code the contact
% information and notecalls; do *not* use \footnote commands.  Let the
% author contact information appear immediately below the author names
% as shown.  We would also prefer that you don't change the type-size
% settings shown here.

\author
{Brandon Olivier, Nicholas Sundin\\
\\
\normalsize{Department of Computer Science, University of Texas at Austin,}\\
\normalsize{2317 Speedway, Stop D9500 Austin, TX 78712}\\
}

% Include the date command, but leave its argument blank.

\date{}



%%%%%%%%%%%%%%%%% END OF PREAMBLE %%%%%%%%%%%%%%%%



\begin{document} 

% Double-space the manuscript.

\baselineskip24pt

% Make the title.

\maketitle 



% Place your abstract within the special {sciabstract} environment.

\begin{sciabstract}
Vocabulary in Twitter posts was used to score the sentiment for areas
in the United States. This score, or happiness index, was aggregated
by zip code using the inherent geographical information present in
some Twitter posts. Conclusions regarding correlation and usefulness
of the happiness index were then gained by comparing it against other
available regional metrics.  Based around this calculated happiness
index, a model was then created that has shown reasonable predictions
of suicide rates in California given an aggregate per-capita income
and happiness index.
\end{sciabstract}

\section*{Introduction}

The idea that our speech conveys more information than just semantic
meaning is an old one.  In Psychopathology of Everyday Life, Sigmund
Freud describes Fehlleistungen, a ``slip of the tongue'', or
``Freudian slip.''\cite{freud} Freud's research into what we say by
mistake has been succeeded by research into how we say what we say.
One may suspect that studying the use of emotionally charged words
would give the best idea of how people are feeling, but research by
James Pennebaker has found that ``pronouns in particular have been
found to strongly correlate with health
improvements.''\cite{pennebaker} The idea that emotionally charged
words are the key to understanding the mental health of speakers is
wrong.  Pennebaker extensively uses a new classification for words:
particles.  A particle is any functional word that requires other
words to derive meaning.  For instance, articles, prepositions, and
conjunctions are particles.  In explaining the importance of
particles, Pennebaker says ``[t]o use a pronoun requires the speaker
and listener to share a common knowledge'', people need to relate and
have an understanding of the other person; we see this same thing
where ``informal settings presuppose a shared frame of reference
\ldots the discerning particle user must have some degree of social
and cognitive skill.''\cite{pennebaker} \\

Much of the speech used in everyday life can be analyzed in terms of
particles. Because ``[i]n the English language there are fewer than
200 commonly used particles, [and] they account for over half of the
words we use.''\cite{pennebaker} Particles are ideal for analyzing the
mood or feelings of someone speaking.  Since much of the content of
our everyday speech is dictated by the situation we are in, merely
analyzing the content of a conversation will not provide as clear of
an insight into the psyche of the speaker.  Particles, on the other
hand, are an element of the style of speech a person is using, and as
such reflect their attitudes at that time, making them ideal units for
sentiment analysis, the use of natural language processing to identify
and extract subjective information in source materials.  \\

Twitter, a social media site limiting posts to 140 characters, was
created in 2006 and immediately surged in popularity.  Twitter boasts
more than 350 million messages, known as tweets, published every day
as of 2012.\cite{twitter} On a cursory glance one would think that
tweets could not be used for much analysis because of their character
limitations. However, based on Pennebaker's The Secret Lives of Pronouns and
other research materials, many believe tweets ``can be combined to
build a larger picture of the user posting them.''\cite{pennebaker}

Microsoft research, in 2013, published a study wherein they collected
data from Twitter and subjects diagnosed with depression.  They used
the tweets from the individuals to train a computer to detect signs of
depression based solely on the person's Twitter feed.  According to
that study, some aspects of a user's feed that may indicate an onset
of depression include: ``decrease in social activity, raised negative
affect, \ldots, heightened relational and
medicinal concerns, and greater expression of religious
involvement.''\cite{microsoft}  

\section*{Method}
Our research builds on the idea that Microsoft Research discusses: we
want to analyze tweets from users on Twitter and assign them a grade
on their mental health based on their tweets.  We can look at tweets
from each user, assess each tweet, then assign a score to the user
based on the content of their tweets.  We then want to aggregate users
based on their zip code, yielding an average happiness index for each
zip code. After that, we will use other data sources, such as weather,
government grants, and suicide rates to try to build a model that will
be able to predict our derived ratio independent of tweets.\\

In 2011, Twitter altered their terms and services and it is now a
violation to share full sets of data.  Ids are still permitted to be
shared, but the actual collection of the full information must be done
on a per use basis.  For our dataset, we got a list of user ids for
twitter users and wrote a python script to gather tweets and save them
in a json object that we could later use.  For each user, we gathered
a collection of tweets, inferred a user's location from the geographic
data encoded into tweets, and created a score for each user.  The
score is based on usage of words.  We used 2 modified versions of
positive and negative words lists that we amended to contain some
internet slang abbreviations and particles.  We split each tweet and compared the
frequency of words from each list as follows:
\begin{quote}
\begin{verbatim}
def calculate_happiness_ratio(tweet_content):
    words = word_counter(tweet_content)
    score = 0
    total_words = 0
    for word, count in words.items():
        if word in positives:
            total_words += count
            score += count
        elif word in negatives:
            total_words += count
    if total_words == 0:
        return None
    else:
        return float(score) / total_words
\end{verbatim}
\end{quote}
where \verb|word_counter| is a function that creates a python
dictionary with each word and the corresponding times that that word
is used in the body of the tweet.  After assigning a score to each
tweet, we reduce to users, and then to zip codes.  \\

Twitter encodes location data as latitude and longitude.  As such, we
had to discover which zip code any arbitrary latitude and longitude
pair falls in.  For that, we turn to MongoDB, which has built in
support for geolocation.  The first step in the geolocation process is
to acquire a file detailing the shapes of every zip code from the US
government, the exact details of getting the shapefiles is available 
in the GitHub for the project.\cite{github}  We used a library called US-Atlas from 
Mike Bostock toget the shapefiles and We used a tool called ogr2ogr 
to convert theshapefiles into the JSON format that MongoDB understands.  
Once the shapes are stored in MongoDB, queries such as seen below.
\begin{quote}
\begin{verbatim}
db.collection.findOne(
  { geometry:
    { $geoIntersects:
      { $geometry :
        { type: "Point",
          coordinates: tweet.coordinates
        }
      }
    }
  }, callbackFunction
);
\end{verbatim}
\end{quote}
Here, the zip code can be handled arbitrarily in a
\verb|callbackFunction|.  Each user was run through a query like this
to assign them a zip code.  After we processed all the data, it was
reduced to a CSV file containing a zip code and a corresponding
happiness score.  \\

To find data to correlate our scores to, we went
to enigma.io, a website devoted to obtaining large datasets and making
them publicly available�� and searched for data indexed by zip
code.  One of the more interesting datasets we used was average income
indexed by zip code.  We plotted the data in Tableau and didn't see a
correlation between income and happiness index.  There are some zip
codes that abide by the maxim ``money doesn't make you happy'', but
others that do not.  They show that if you make more money, you are
more likely to be happy.\\

In addition to these methods, we used SQL developer and Oracle Data
Miner to to process happiness index, suicide rate, and income columns
using the decision tree, naive Bayes, and SVM (support vector
machines) to analyze our data.  We found no statistically significant
link between the three columns using these methods.  \\

\section*{Results \&\ Discussion}

We created a histogram to see the suicide rate of California zip codes
plotted against their corresponding happiness index.  We obtained the
suicide rate by dividing the number of suicides that occur in a given
zipcode, as provided by the California statistics, and we divide that
by the total number of deaths in a zipcode.  As the histogram shows, there
is a correlation between happiness and suicide: as the happiness index
decreases, so does the suicide rate for that zip code.
\begin{figure}[ht]
\centering
\includegraphics[width=1\textwidth]{figure1}
\caption{Suicide vs. Happiness}
\end{figure}
\\

To resolve some discrepancies, we modified what we were looking at.
In figure 2, we divide the average income by the happiness index to
offset the effects of poverty on mental health.  That yields a value
which represents the income per happiness unit. We call this a
happiness normalized income (HNI).  So if a person lives in place A,
with an HNI of 120,000, and is considering moving to a place with an
HNI of 100,000, then to maintain their level of happiness, they only
need to make \$100,000 per year, as opposed to \$120,000.  Thus HNI is a
value that represents how much money one needs to maintain a certain
level of happiness.  We then plot the suicide rate by the HNI
in a new histogram.  The suicide rates in figure two are aggregated by average and collected into bins 
for display purposes. The points in figure 2 are colored to represent the happiness
ratio.  \\

\begin{figure}[ht]
\centering
\includegraphics[width=1\textwidth]{figure2}
\caption{Suicide vs. Happiness Normalized Income (HNI)}
\end{figure}
There is a correlation between suicide rate and HNI.  Figure 2 shows that if 
one requires more money to be happy in a
particular area, then that area has fewer suicides.  For instance, an area with
 a HNI value of 80K should have an higher suicide rate than a zip code
that has an HNI of 100K.\\

We can conclude from these graphs that people in low income areas and
areas that have higher measured happiness indices are more likely to
commit suicide.  While it may seem contradictory that areas with
higher measured happiness have a higher incidence of suicide, it would
make sense if online  posts could be considered a healthy venting of
frustration.  \\

We then used this correlation to make predictions
about the suicide rate in different areas.  Rather than make
complicated formulas, we elected to use a simple linear fit line to
make predictions.  It will be less accurate for many values, but it
should still give overall correct results.
\begin{figure}[ht]
\centering
\includegraphics[width=1\textwidth]{figure3}
\caption{Predicted vs. Actual Suicide Rate}
\end{figure}
Many of the predictions are muted in the middle of the spectrum, but
on the whole, it seems to give a decent prediction of suicide rates in
different zip codes in California.  \\

We also found an unusual trend in a graph comparing happiness to
income.  For incomes less than \$800K, the happiness index is
relatively flat; people are just as happy whether they make \$80K or
\$200K.  However, at \$800K yearly income per capita (per person in
the household), something
\begin{figure}[ht]
\centering
\includegraphics[height=0.85\textwidth]{figure4}
\caption{histogram of nationwide happiness by income}
\end{figure}
interesting happens: the happiness index plummets, then recovers.  One
of the more interesting things about the trend is that only one of the
bars past \$800K is only one zip code (the highest bar is just 10023,
a zip code corresponding to an area in the middle of Manhattan, near
Central Park and the Upper West Side).  Since the others all include
multiple zip codes, we believe that they are not outliers or mistakes.
\\

We predict two possible causes for that.  One potential cause is that
once a person reaches \$800K in yearly salary, they enter a new social
circle that values money much higher than their previous social
circle.  That makes them less happy, if they are dependent on their
money for happiness.  They are now at the bottom of their social
circle in terms of monetary wealth.  \\

Another potential cause is that there is some single area that is an
outlier and is requiring a higher income for happiness.  We believe
this to be an unlikely cause because it would be very odd that there
is some area where one is affected by only earning \$800K.  That is a
lot of money.  To test that hypothesis, we filtered the happiness
ratio to exclude some areas in New York, as that's the most expensive
place to live in the United States. When excluding New York, one gets
the following graph.
\begin{figure}[ht]
\centering
\includegraphics[height=0.75\textwidth]{figure5}
\caption{Nationwide happiness by income excluding New York}
\end{figure}
\\

Since this graph also shows the same dip at \$800K, we do not
believe that there is a single area that has such a high cost of
living that \$800K yearly income isn't sufficient.  It is possible
that there is some data loss caused by aggregation of income for an
entire zip code.  For instance, if Warren Buffet moved to an otherwise
very low income zip code, their yearly income would spike, but that is
unlikely for the same reason: there are many zip codes that exhibit
this behavior.\\

\section*{Conclusion}
Although there does seem to be some correlation between income and
happiness, the exact nature of it appears complex and non-linear. Much
stronger correlations were found between suicide rate and happiness
ratio. A paradoxical correlation shows that zip codes of negative
sentiment trend toward fewer suicides. The happiness index in
conjunction with the income data reveals a stronger trend-line for our
suicide rate findings, suggesting that the happiness index alone is
not enough to predict social trends and temperament. \\


\section*{Acknowledgements}
We would like to thank Dr. Phillip Cannata at the University of Texas
at Austin for his mentorship and support of our research as well as
the use of his Oracle database and computing cloud. \\

Our code is publicly available on GitHub.\cite{github}
\pagebreak

\begin{thebibliography}{9}

\bibitem{pennebaker}
  Campbell, R. S., and J. W. Pennebaker,
  \emph{The Secret Life Of Pronouns: Flexibility in Writing Style and Physical Health},
  Psychological Science 14.1 (2003),
  60--65,
  Web.

\bibitem{freud}
  Freud, Sigmund. 
  Psychopathology Of Everyday Life
  1949. 
  Print.

\bibitem{atlas} 
  US Atlas 
  \url{https://github.com/mbostock/us-atlas}

\bibitem{ogr} 
  Geospatial Data Abstraction Library
  \url{https://www.npmjs.com/package/ogr2ogr}

\bibitem{twitter} 
  Twitter Turns Six
  \url{https://blog.twitter.com/2012/twitter-turns-six}


\bibitem{microsoft} 
  Microsoft Depression Research 
  \url{http://research.microsoft.com/apps/pubs/default.aspx?id=192721}

\bibitem{neal} 
  Neal Caren
  Positive and Negative Word Lists
  \url{http://www.unc.edu/~ncaren/haphazard/negative.txt}
  \url{http://www.unc.edu/~ncaren/haphazard/positive.txt}
  
\bibitem{github}
  Public Github with source code
  \url{https://github.com/bolivier/370Research}
  
\end{thebibliography}



\end{document}
